% !TeX TS-program = xelatex
% !TeX builder = latexmk
% !BIB program = biblatex

% XeLaTeX stuff:
% Normalize any residual Unicode combining accents and write out error messages
\XeTeXinputnormalization=1
\tracinglostchars=1
\tracingonline=1
\XeTeXgenerateactualtext=1

% Don't modify the `\DocumentMetadata` command unless you know what it does!
% If this command throws an "undefined" error, your LaTeX system is out of date.
\DocumentMetadata{
  pdfstandard = a-2b,
  pdfversion  = 1.7,
  lang		    = en-US
}

\documentclass{xminimal}

\title{\relax}
\author{\relax}
\date{\relax}

\begin{document}
\thispagestyle{empty}

The \sa{Upaniṣad} being mainly intended for a knowledge of its meaning, there
should be no want of care in the study of the text. Therefore here follows
a lesson on \tl{Śikṣā}, the doctrine of pronunciation.

\begin{verse}\centering

  \dn{ॐ शीक्षां व्याख्यास्यामः ।\\ वर्णः स्वरः । मात्रा बलम् । साम सन्तानः ।\\ इत्युक्तः शीक्षाध्यायः ॥}

  \sa{\textbf{oṃ śīkṣāṃ vyākhyāsyāmaḥ |\\ varṇaḥ svaraḥ | mātrā balam | sāma santānaḥ |\\ ityuktaḥ śīkṣādhyāyaḥ ||}}

\end{verse}

\begin{quote}
  \emph{Om! We shall treat of the phonetics: sound, rhythm, quantity, strength,
    mo\-dulation, union. Thus has been declared the lesson on phonetics.}
\end{quote}

Phonetics (\tl{śikṣā}) is the science which treats of sounds and their
pronunciation. Or, the word ‘\tl{śikṣā}’ may here signify the sounds \etc
which are treated of in that science. Sound: such as ‘ā’. Rhythm: such as
\tl{udātta} or high-pitched tone. Length: short, long, \etc  Strength: intensity
of effort. Modulation: pronunciation of sounds in the middle tone. Union:
conjunction of several sounds.—These are the things to be learnt. Thus far is
the lesson on phonetics, in these words the \sa{Upaniṣad} concludes the present
subject with a view to proceed to the next.

For him who, by the recitation of the mantra given in the first \tl{anuvāka},
has removed obstacles, it is proper to proceed with the text treating of
the ways of contemplation and of the nature of Brahman. As the text of the
\sa{Upaniṣad} is mainly intended for a knowledge of the things therein treated
of, one should spare no pains in learning the text; and accordingly the
\sa{Upaniṣad} proceeds with a lesson on phonetics. Here one may ask, what if one
be careless? We reply: carelessness will lead to evil. It has been said,
“The mantra, when wanting in rhythm or sound, or when wrongly used, conveys not
the intended idea. That thunderbolt of speech will ruin the worshipper as
the word ‘\tl{indra-śatru}’ did owing to a fault in rhythm”.

(Objection:)—If so, this lesson should have been given in the \tl{karma-kāṇḍa}
or ritualistic section.

(Answer:)—True. For that very reason,—as the lesson subserves both the sections,
—it is given between the two sections.

(Objection:)—Then, as subservient to both, let it be given at the beginning of
the Veda.

(Answer:)—Though subservient to both, it has to be given in the theosophical
section in order to shew its greater use as regards knowledge. As to
the ritualistic section, despite the chance of misunderstanding the scriptures
owing to error in the rhythm and sound, it is possible to do away with any
imperfection in the performance by \tl{prāyaśchitta} or an expiatory act.
Accordingly, in such cases, the Veda gives the following mantra for an expiatory
offering of clarified butter:

\begin{quote}
  “Whatever in the sacrifice is wrongly done, unknown or known, do, O Agni,
  rectify that (part) of this (sacrifice); thou indeed knowest what is right.”
\end{quote}

On the contrary, when the scriptures in the theosophical section are wrongly
understood, the imperfection cannot be made up for. Indeed, it is not possible
to do away with wrong knowledge by an expiatory act. We have never seen
an illusory perception of serpent in a rope removed by the reciting of
the \sa{Gāyatrī} hymn. Wherefore no expiatory act whatever is enjoined in
connection with knowledge, in the same way that it is enjoined in connection
with the rituals. On the contrary, in the case of him who, striving in the path
of wisdom commits any sin, the scriptures deny all expiation other than
theosophy, in the following words:

\begin{quote}
  “If the yogin should unguardedly commit a sin, he should resort to yoga alone,
  never to any other thing such as mantra.”
\end{quote}

Wherefore the lesson on the phonetics is given here especially to enjoin great
care in the study of the \sa{Upaniṣads}, so that there may be no defect in
the knowledge acquired and that the scripture may be understood aright.

\end{document}
