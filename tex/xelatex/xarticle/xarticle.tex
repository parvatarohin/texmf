% !TeX TS-program = xelatex
% !TeX builder = latexmk
% !BIB program = biblatex

% XeLaTeX stuff:
% Normalize any residual Unicode combining accents and write out error messages
\XeTeXinputnormalization=1
\tracinglostchars=1
\tracingonline=1
\XeTeXgenerateactualtext=1

% Don't modify the `\DocumentMetadata` command unless you know what it does!
% If this command throws an "undefined" error, your LaTeX system is out of date.
\DocumentMetadata{
  pdfstandard = a-2b,
  pdfversion  = 1.7,
  lang		    = en-US
}

\documentclass[twoside]{xarticle}

\setcounter{tocdepth}{2}
\setcounter{secnumdepth}{0}

\title{\relax}
\author{\relax}
\date{\relax}

\hypersetup{%
  pdftitle={\relax},
  pdfauthor={\relax},
  pdfauthortitle={Prof.},
	pdfsubject={XeLaTeX},
	pdfkeywords={latex,xelatex,typesetting}
}

\begin{document}
\hyphenation{%
  deva-nāgarī
  deva-nagari
  śaṅ-karā-cārya
  san-s-krit
}

\thispagestyle{empty}
\maketitle
\tableofcontents

\clearpage

\pagestyle{fancy}

\section{Introduction}
\thispagestyle{empty}

Traditionally, text is composed to create a readable, coherent, and visually
satisfying typeface that works invisibly, without the awareness of the reader.
Even distribution of typeset material, with a minimum of distractions and
anomalies, is aimed at producing clarity and transparency. Choice of typeface(s)
is the primary aspect of text typography—prose fiction, non-fiction, editorial,
educational, religious, scientific, spi\-ritual, and commercial writing all have
differing characteristics and requirements of appropriate typefaces and their
fonts or styles.

\clearpage

\thispagestyle{empty}

\section{Section}
\subsection{Sub Section}
\subsubsection{Sub Sub Section}

\clearpage

\section{New Section}
\subsection{Sub Section}
\subsubsection{Sub Sub Section}

\clearpage

\appendix

\section{The Appendix}
\thispagestyle{empty}

Traditionally, text is composed to create a readable, coherent, and visually
satisfying typeface that works invisibly, without the awareness of the reader.
Even distribution of typeset material, with a minimum of distractions and
anomalies, is aimed at producing clarity and transparency. Choice of typeface(s)
is the primary aspect of text typography—prose fiction, non-fiction, editorial,
educational, religious, scientific, spiritual, and commercial writing all have
differing characteristics and requirements of appropriate typefaces and their
fonts or styles.

\end{document}
